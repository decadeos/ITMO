\documentclass[a4paper,12pt]{article}
\usepackage[utf8]{inputenc}
\usepackage[english,russian]{babel}
\usepackage{geometry}
\geometry{top=2cm, bottom=2cm, left=3cm, right=3cm}
\documentclass[a4paper,12pt]{article}
\usepackage{listings} % Подключаем пакет для кода
\usepackage{xcolor}   % Для цветного форматирования
\usepackage{geometry}
\geometry{top=2cm, bottom=2cm, left=2cm, right=2cm}
\usepackage{setspace}
\onehalfspacing
\usepackage{amsmath,amsfonts,amssymb}
\usepackage{graphicx}
\usepackage[none]{hyphenat}
\usepackage{hyperref} % Пакет для гиперссылок
% Настройка отображения кода
\lstset{
    language=Python,           % Указываем язык
    basicstyle=\ttfamily\small, % Шрифт кода
    keywordstyle=\color{blue}, % Цвет ключевых слов
    stringstyle=\color{red},   % Цвет строк (например, "текст")
    commentstyle=\color{green},% Цвет комментариев
    numbers=left,              % Нумерация строк
    numberstyle=\tiny\color{gray}, % Формат номеров строк
    stepnumber=1,              % Шаг нумерации строк
    breaklines=true,           % Перенос строк
    backgroundcolor=\color{lightgray!10}, % Цвет фона
    frame=single,              % Рамка вокруг кода
    tabsize=4,                 % Размер табуляции
    showstringspaces=false     % Не отображать пробелы в строках
}



\begin{document}

\begin{titlepage}
    \centering
    {\scshape\large Министерство науки и высшего образования Российской Федерации\par}
    {\scshape\large федеральное государственное автономное образовательное учреждение высшего образования\par}
    {\bfseries\large «НАЦИОНАЛЬНЫЙ ИССЛЕДОВАТЕЛЬСКИЙ УНИВЕРСИТЕТ ИТМО»\par}
    \vspace{3cm}
    {\bfseries\Large РГР 1 по Дифферинциальным уравнениям\par}
    \vspace{1cm}
    {\bfseries\large «ЧИСЛЕННОЕ
ИНТЕГРИРОВАНИЕ
ДИФФЕРЕНЦИАЛЬНЫХ
УРАВНЕНИЙ ПЕРВОГО
ПОРЯДКА»\par}
    \vfill
    \hspace{0.5\linewidth}%
    \begin{minipage}{0.4\linewidth}
        Выполнилa: \par Охрименко А. Д. \par
        R3225, 409290 \par
        Поток: ДУ 23 1.1 \par
        Факультет: СУиР \par
        Преподаватель: \par
        Танченко Ю. В.
    \end{minipage}
    \vfill
    {\bfseries Санкт-Петербург, 2024\par}
\end{titlepage}






\section{Вариант 20. Условие задачи}

Численно решить дифференциальное уравнение:
\[
y' = 3x - \frac{y}{x}, \quad y(1) = 1
\]
на отрезке $[1; 2]$ с шагом $h = 0.2$ методом Эйлера, модифицированным методом Эйлера.

Найти точное решение $y = y(x)$ и сравнить значения точного и приближенных решений в точке $x = 2$. Найти абсолютную и относительную погрешности в этой точке для каждого метода. Вычисления вести с четырьмя десятичными знаками.

\textbf{Замечание.} Необходимые теоретические сведения и образец выполнения работы приведены в конце пункта.



\section{Точное решение}

\subsection*{Приведение уравнения к однородному виду}
Дано дифференциальное уравнение:
\[
y' = 3x - \frac{y}{x}.
\]

Запишем его в форме:
\[
\frac{dy}{dx} = 3x - \frac{y}{x}.
\]

Предположим подстановку \( y = z^2 \), где \( z = \sqrt{y} \). Тогда:
\[
y = z^2, \quad \frac{dy}{dx} = 2z \frac{dz}{dx}.
\]

Подставляем в уравнение:
\[
2z \frac{dz}{dx} = 3x - \frac{z^2}{x}.
\]

Приводим к однородному виду:
\[
2z \frac{dz}{dx} = \frac{3x^2 - z^2}{x}.
\]

\subsection*{Разделение переменных}
Перепишем уравнение в форме, удобной для разделения переменных:
\[
2z \, dz = \frac{3x^2 - z^2}{x} \, dx.
\]

Разделим переменные:
\[
\frac{2z}{z^2 - 3} \, dz = \frac{1}{x} \, dx.
\]

Интегрируем обе части:
\[
\int \frac{2z}{z^2 - 3} \, dz = \int \frac{1}{x} \, dx.
\]

\subsection*{Решение интегралов}
Интегрируем левую часть:
\[
\int \frac{2z}{z^2 - 3} \, dz = \ln|z^2 - 3| + C_1.
\]

Интегрируем правую часть:
\[
\int \frac{1}{x} \, dx = \ln|x| + C_2.
\]

Собираем итоговое выражение:
\[
\ln|z^2 - 3| = \ln|x| + C,
\]
где \( C = C_2 - C_1 \).

Возвращаемся к переменной \( y \) через \( z \):
\[
z^2 - 3 = Cx.
\]

Так как \( z = \sqrt{y} \), то:
\[
y = x^2 + Cx.
\]

\subsection*{Задача Коши}
Учитываем начальное условие \( y(1) = 1 \):
\[
1 = 1^2 + C \cdot 1.
\]

Решаем относительно \( C \):
\[
C = 0.
\]

\subsection*{Итоговое решение}
Подставляем значение \( C = 0 \) в общее решение:
\[
y = x^2.
\]

Таким образом, точное решение задачи Коши:
\[
y = x^2.
\]


\section{Метод Эйлера}



Рассмотрим задачу Коши:
\[
y'(x) = 3x - \frac{y}{x}, \quad y(1) = 1.
\]

Шаг интегрирования: \( h = 0.2 \). Необходимо найти приближённые значения \( y \) на отрезке \([1; 2]\) методом Эйлера.

\subsection*{Формула метода Эйлера}
Приближённые значения вычисляются по формуле:
\[
y_{n+1} = y_n + h \cdot f(x_n, y_n),
\]
где \( f(x, y) = 3x - \frac{y}{x} \).

\subsection*{Пошаговые вычисления}

\textbf{Шаг 0:} 
\[
x_0 = 1, \quad y_0 = 1.
\]

\textbf{Шаг 1:}
\[
x_1 = x_0 + h = 1 + 0.2 = 1.2,
\]
\[
f(x_0, y_0) = 3 \cdot 1 - \frac{1}{1} = 3 - 1 = 2,
\]
\[
y_1 = y_0 + h \cdot f(x_0, y_0) = 1 + 0.2 \cdot 2 = 1.4.
\]

\textbf{Шаг 2:}
\[
x_2 = x_1 + h = 1.2 + 0.2 = 1.4,
\]
\[
f(x_1, y_1) = 3 \cdot 1.2 - \frac{1.4}{1.2} = 3.6 - 1.1667 \approx 2.4333,
\]
\[
y_2 = y_1 + h \cdot f(x_1, y_1) = 1.4 + 0.2 \cdot 2.4333 = 1.8867.
\]

\textbf{Шаг 3:}
\[
x_3 = x_2 + h = 1.4 + 0.2 = 1.6,
\]
\[
f(x_2, y_2) = 3 \cdot 1.4 - \frac{1.8867}{1.4} = 4.2 - 1.3476 \approx 2.8524,
\]
\[
y_3 = y_2 + h \cdot f(x_2, y_2) = 1.8867 + 0.2 \cdot 2.8524 = 2.4572.
\]

\textbf{Шаг 4:}
\[
x_4 = x_3 + h = 1.6 + 0.2 = 1.8,
\]
\[
f(x_3, y_3) = 3 \cdot 1.6 - \frac{2.4572}{1.6} = 4.8 - 1.5358 \approx 3.2642,
\]
\[
y_4 = y_3 + h \cdot f(x_3, y_3) = 2.4572 + 0.2 \cdot 3.2642 = 3.1100.
\]

\textbf{Шаг 5:}
\[
x_5 = x_4 + h = 1.8 + 0.2 = 2.0,
\]
\[
f(x_4, y_4) = 3 \cdot 1.8 - \frac{3.1100}{1.8} = 5.4 - 1.7278 \approx 3.6722,
\]
\[
y_5 = y_4 + h \cdot f(x_4, y_4) = 3.1100 + 0.2 \cdot 3.6722 = 3.8444.
\]

\subsection*{Итоговая таблица значений}

\[
\begin{array}{|c|c|c|c|}
\hline
n & x_n & f(x_n, y_n) & y_n \\ \hline
0 & 1.0 & 2.0 & 1.0000 \\ 
1 & 1.2 & 2.4333 & 1.4000 \\ 
2 & 1.4 & 2.8524 & 1.8867 \\ 
3 & 1.6 & 3.2642 & 2.4572 \\ 
4 & 1.8 & 3.6722 & 3.1100 \\ 
5 & 2.0 & - & 3.8444 \\ \hline
\end{array}
\]

\subsection*{Вывод}
При использовании метода Эйлера для решения задачи Коши приближённое значение решения в точке \( x = 2 \) равно \( y = 3.8444 \).



\subsection*{Код}
\par
Ниже представлен код метода Эйлера для решения задачи:

\begin{lstlisting}
import numpy as np
import pandas as pd

def f(x, y):
    return 3 * x - y / x

def euler_method(f, x0, y0, h, x_end):
    x_values = [x0]
    y_values = [y0]
    
    x = x0
    y = y0
    
    while x < x_end:
        y = y + h * f(x, y)
        x = x + h
        x_values.append(round(x, 4))  
        y_values.append(round(y, 4)) 
    
    return np.array(x_values), np.array(y_values)

x0 = 1   
y0 = 1    
h = 0.2   
x_end = 2 

x_vals, y_vals = euler_method(f, x0, y0, h, x_end)

results = pd.DataFrame({'x': x_vals, 'y (Eiler)': y_vals})
print(results)
\end{lstlisting}






















\section{Модифицированный метод Эйлера}


Рассмотрим задачу Коши:
\[
y'(x) = 3x - \frac{y}{x}, \quad y(1) = 1.
\]

Шаг интегрирования: \( h = 0.2 \). Необходимо найти приближённые значения \( y \) на отрезке \([1; 2]\) модифицированным методом Эйлера.

\subsection*{Формула метода}
1. Промежуточное значение (предиктор):
\[
y^* = y_n + h \cdot f(x_n, y_n).
\]
2. Коррекция (корректор):
\[
y_{n+1} = y_n + \frac{h}{2} \cdot \left[f(x_n, y_n) + f(x_{n+1}, y^*)\right].
\]

\subsection*{Пошаговые вычисления}

\textbf{Шаг 0:}
\[
x_0 = 1, \quad y_0 = 1.
\]

\textbf{Шаг 1:}
\[
x_1 = x_0 + h = 1.2, \quad f(x_0, y_0) = 2.0,
\]
\[
y^* = y_0 + h \cdot f(x_0, y_0) = 1 + 0.2 \cdot 2.0 = 1.4,
\]
\[
f(x_1, y^*) = 3 \cdot 1.2 - \frac{1.4}{1.2} = 2.4333,
\]
\[
y_1 = y_0 + \frac{h}{2} \cdot \left[f(x_0, y_0) + f(x_1, y^*)\right] = 1 + 0.1 \cdot (2.0 + 2.4333) = 1.4433.
\]

\textbf{Шаг 2:}
\[
x_2 = x_1 + h = 1.4, \quad f(x_1, y_1) = 2.3461,
\]
\[
y^* = y_1 + h \cdot f(x_1, y_1) = 1.4433 + 0.2 \cdot 2.3461 = 1.9125,
\]
\[
f(x_2, y^*) = 3 \cdot 1.4 - \frac{1.9125}{1.4} = 2.8524,
\]
\[
y_2 = y_1 + \frac{h}{2} \cdot \left[f(x_1, y_1) + f(x_2, y^*)\right] = 1.4433 + 0.1 \cdot (2.3461 + 2.8524) = 1.9632.
\]

\subsection*{Итоговая таблица значений}

\[
\begin{array}{|c|c|c|c|c|}
\hline
n & x_n & y^* & f(x_n, y_n) & y_n \\ \hline
0 & 1.0 & - & 2.0000 & 1.0000 \\ 
1 & 1.2 & 1.4000 & 2.4333 & 1.4433 \\ 
2 & 1.4 & 1.9125 & 2.8524 & 1.9657 \\ 
3 & 1.6 & 2.5031 & 3.2642 & 2.5675 \\ 
4 & 1.8 & 3.1615 & 3.6722 & 3.2488 \\ 
5 & 2.0 & - & - & 4.0100 \\ \hline
\end{array}
\]


\subsection*{Код}

\begin{lstlisting}
import numpy as np
import pandas as pd

def f(x, y):
    return 3 * x - y / x

def modified_euler_method(f, x0, y0, h, x_end):
    x_values = [x0]
    y_values = [y0]
    
    x = x0
    y = y0
    
    while x < x_end:
        y_pred = y + h * f(x, y)   
        x_next = x + h           
        y = y + (h / 2) * (f(x, y) + f(x_next, y_pred))  
        x = x_next
        
        x_values.append(round(x, 4))  
        y_values.append(round(y, 4)) 
    
    return np.array(x_values), np.array(y_values)

x0 = 1    
y0 = 1    
h = 0.2   
x_end = 2

x_vals, y_vals = modified_euler_method(f, x0, y0, h, x_end)

results = pd.DataFrame({'x': x_vals, 'y (Mod. Eiler)': y_vals})
print("Results:")
print(results)

\end{lstlisting}
\subsection*{Вывод}
При использовании модифицированного метода Эйлера приближённое значение решения в точке \( x = 2 \) равно \( y = 3.9547 \).








\section{Сравнение точного решения и приближённых решений}

\begin{table}[h!]
\centering

\begin{tabular}{|c|c|c|c|c|c|c|c|}
\hline
Решение & \( x = 1.2 \) & \( x = 1.4 \) & \( x = 1.6 \) & \( x = 1.8 \) & \( x = 2.0 \) & Абс. погрешн. & Относ. погрешн. \\ \hline
Точное & 1.4400 & 1.9600 & 2.5600 & 3.2400 & 4.0000 & - & - \\ \hline
Эйлер & 1.4000 & 1.8867 & 2.4572 & 3.1100 & 3.8444 & 0.1556 & 3.89\% \\ \hline
Мод. Эйлер & 1.4433 & 1.9657 & 2.5675 & 3.2488 & 4.0100 & 0.0453 & 0.25\% \\ \hline 
\end{tabular}
\end{table}

\end{document}

